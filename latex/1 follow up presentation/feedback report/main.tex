

\documentclass[letterpaper, 10 pt, conference]{ieeeconf}  

\IEEEoverridecommandlockouts                              
\overrideIEEEmargins


\title{\LARGE \bf
Developing a Benchmarking Framework for RAG Systems \newline Feedback Report
}


\author{By Jan Albrecht}


\begin{document}



\maketitle
\thispagestyle{plain}
\pagestyle{plain}



%%%%%%%%%%%%%%%%%%%%%%%% INTRODUCTION - DONE %%%%%%%%%%%%%%%%%%%%%%%%%%%%%%%%%%%%%%%%
\section{Zusammenfassung der Feedbacks}
% Gehe hier auf alle Feedbacks im Allgemeinen ein und motiviere sie. 
% Gehe in den nachfolgenden Sektionen auf die einzelnen Feedbacks ein.
% - Mehr Fokus auf Implementierung der Re-Konfigurierbarkeit
%   - Mehr Metriken (System / HW / Andere? -> siehe AISE)
% - Weniger Fokus auf das eigentliche Benchmarken

%%%%%%%%%%%%%%%%% .CREATING A .TEX FILE - DONE %%%%%%%%%%%%%%%%%%%%%%%%%%%%%%

\section{Angepasste Umsetzung des Frameworks}
% Mehr Fokus auf Implementierung der Re-Konfigurierbarkeit
% Es müssen mehr Metriken zur Verfügung stehen -> Wie zum Beispiel die Inferenzzeit für eine Anfrage.
% -> Daraus resultiert allerdings auch, dass Prozesse parallelisiert gestartet werden müssen. 
%   -> Welche könenn denn parallel gestartet werden?
%   -> Wie müssen die Prozesse experimentell ablaufen, sodass keine anderen Prozesse diese beeinflussen (interne Validität)
%   -> Daraus folgen Hardware Herausforderungen, welche ich ohne Sebastian Simon nicht schaffen kann, da ich die Univerisitäts Infrastruktur nicht kenne

\section{Angepasste Umsetzung der Benchmarks}
% Replikation von bereits durchgeführten Experimenten

\section{How to: References}

% Demonstrate how to set up the 	thebibliography environment
%  Explain how to include each bibitem

% Explain the following commands:
% \label
% \ref
% \cite
% Note: It will be helpful to use a previous figure as an example for \label and \ref

%%%%%%%%%%%%%%%%%%%%%%%%%%%%%%%%%%%%%%%%%%%%%%%%%%%%%%%%%%%%%%%%%%%%%%%%%%%%%%%%

\section*{ACKNOWLEDGMENT}

- Overleaf.com, for the template



% For acknowledgements, thank anyone who personally helped you with this assignment.
% If you read a general tutorial or watched an instructional YouTube video on LaTeX, but didn’t specifically reference it, thank the creators.
% If you received help from a current CMPE 185 student, let us know.  

% Anything you do specifically reference should, of course, be cited and included in the References section.


%%%%%%%%%%%%%%%%%%%%%%%%%%%%%%%%%%%%%%%%%%%%%%%%%%%%%%%%%%%%%%%%%%%%%%%%%%%%%%%%

\section*{APPENDIX}

Appendixes should appear before the acknowledgment. 
\newline
\newline
- Any sources?

Overleaf.com

References are important to the reader; therefore, each citation must be complete and correct. If at all possible, references should be commonly available publications.






\end{document}

% \begin{itemize}

% \item Use either SI (MKS) or CGS as primary units. (SI units are encouraged.) English units may be used as secondary units (in parentheses). An exception would be the use of English units as identifiers in trade, such as Ò3.5-inch disk driveÓ.
% \item Avoid combining SI and CGS units, such as current in amperes and magnetic field in oersteds. This often leads to confusion because equations do not balance dimensionally. If you must use mixed units, clearly state the units for each quantity that you use in an equation.
% \item Do not mix complete spellings and abbreviations of units: ÒWb/m2Ó or Òwebers per square meterÓ, not Òwebers/m2Ó.  Spell out units when they appear in text: Ò. . . a few henriesÓ, not Ò. . . a few HÓ.
% \item Use a zero before decimal points: Ò0.25Ó, not Ò.25Ó. Use Òcm3Ó, not ÒccÓ. (bullet list)

% \end{itemize}




% \subsection{Equations}

% The equations are an exception to the prescribed specifications of this template. You will need to determine whether or not your equation should be typed using either the Times New Roman or the Symbol font (please no other font). To create multileveled equations, it may be necessary to treat the equation as a graphic and insert it into the text after your paper is styled. Number equations consecutively. Equation numbers, within parentheses, are to position flush right, as in (1), using a right tab stop. To make your equations more compact, you may use the solidus ( / ), the exp function, or appropriate exponents. Italicize Roman symbols for quantities and variables, but not Greek symbols. Use a long dash rather than a hyphen for a minus sign. Punctuate equations with commas or periods when they are part of a sentence, as in

% $$
% \alpha + \beta = \chi \eqno{(1)}
% $$

% Note that the equation is centered using a center tab stop. Be sure that the symbols in your equation have been defined before or immediately following the equation. Use Ò(1)Ó, not ÒEq. (1)Ó or Òequation (1)Ó, except at the beginning of a sentence: ÒEquation (1) is . . .





\section{USING THE TEMPLATE}

Use this sample document as your LaTeX source file to create your document. Save this file as {\bf root.tex}. You have to make sure to use the cls file that came with this distribution. If you use a different style file, you cannot expect to get required margins. Note also that when you are creating your out PDF file, the source file is only part of the equation. {\it Your \TeX\ $\rightarrow$ PDF filter determines the output file size. Even if you make all the specifications to output a letter file in the source - if you filter is set to produce A4, you will only get A4 output. }

It is impossible to account for all possible situation, one would encounter using \TeX. If you are using multiple \TeX\ files you must make sure that the ``MAIN`` source file is called root.tex - this is particularly important if your conference is using PaperPlaza's built in \TeX\ to PDF conversion tool.

\subsection{Headings, etc}

Text heads organize the topics on a relational, hierarchical basis. For example, the paper title is the primary text head because all subsequent material relates and elaborates on this one topic. If there are two or more sub-topics, the next level head (uppercase Roman numerals) should be used and, conversely, if there are not at least two sub-topics, then no subheads should be introduced. Styles named ÒHeading 1Ó, ÒHeading 2Ó, ÒHeading 3Ó, and ÒHeading 4Ó are prescribed.

\subsection{Figures and Tables}

Positioning Figures and Tables: Place figures and tables at the top and bottom of columns. Avoid placing them in the middle of columns. Large figures and tables may span across both columns. Figure captions should be below the figures; table heads should appear above the tables. Insert figures and tables after they are cited in the text. Use the abbreviation ÒFig. 1Ó, even at the beginning of a sentence.

\begin{table}[h]
\caption{An Example of a Table}
\label{table_example}
\begin{center}
\begin{tabular}{|c||c|}
\hline
One & Two\\
\hline
Three & Four\\
\hline
\end{tabular}
\end{center}
\end{table}


   \begin{figure}[thpb]
      \centering
      \framebox{\parbox{3in}{We suggest that you use a text box to insert a graphic (which is ideally a 300 dpi TIFF or EPS file, with all fonts embedded) because, in an document, this method is somewhat more stable than directly inserting a picture.
}}
      %\includegraphics[scale=1.0]{figurefile}
      \caption{Inductance of oscillation winding on amorphous
       magnetic core versus DC bias magnetic field}
      \label{figurelabel}
   \end{figure}
   