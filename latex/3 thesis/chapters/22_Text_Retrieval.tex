
There are several types of text retrieval for RAG systems. \cite{Zhao.29.02.2024} classified four different types of retrieval techniques regarding how the retrieved information are passed to the generation. In this thesis I will only focus on query-based retrievals as they are the most common and widely used retrieval techniques. The other three types are latent-representative-based retrieval, logit-based and speculative retrieval.

Another categorization of text retrieval can be done by the type of retrieval. There are three main types of retrieval: sparse retrieval, dense retrieval and hybrid retrieval. Sparse retrieval is based on traditional information retrieval techniques like TF-IDF or BM25. Dense retrieval is based on neural networks and embeddings like BERT or DPR. Hybrid retrieval combines both sparse and dense retrieval techniques. Following I
 will describe the most common retrieval techniques for each type of retrieval.

\subsection{Sparse Retrieval}
\label{sec:sparse_retrieval}

\paragraph{TF-IDF}
\label{sec:tfidf}

The TF-IDF algorithm (term frequency - inverse document frequency) belongs to the 

\paragraph{BM25}
\label{sec:bm25}

\subsection{Dense Retrieval}
\label{sec:dense_retrieval}

\paragraph{BERT}
\label{sec:bert}

\paragraph{DPR}
\label{sec:dpr}

\subsection{Hybrid Retrieval}
\label{sec:hybrid_retrieval}