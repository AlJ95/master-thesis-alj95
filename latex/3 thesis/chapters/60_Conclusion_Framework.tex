%% SOLL NOCH REIN:
% Mit mehr Zeit und mehr finanziellen Möglichkeiten kann das Framework benutzt werden um eine Vielzahl von Embedding Models oder anderen RAG confgiurations getestet werden. Die Zeit und finanziellen Mittel haben hier nicht ausgereicht, um dies zu schaffen. Insgesamt haben sich die Kosten der Experimente auf ca. 200€ belaufen, welches ein geringer Wert für industrielle Maßstäbe ist.

The advent of Large Language Models (LLMs) has revolutionized natural language processing, yet their inherent limitations, such as knowledge cutoffs, potential for hallucination, and difficulty in accessing private or rapidly changing information, necessitate more robust solutions for many real-world applications. Retrieval-Augmented Generation (RAG) systems have emerged as a promising approach to mitigate these issues by grounding LLM responses in external knowledge sources. However, as explored throughout this thesis, this approach is not without its own significant trade-offs. The integration of retrieval mechanisms introduces significant complexities, increased inference times, and new challenges in evaluation and optimization.

This thesis confronted these challenges by addressing two central questions: "Is an advanced RAG system necessary for a given use case, or is a standalone LLM sufficient?" and "How can one determine if a specific RAG configuration is optimal for its intended task?" The primary contribution of this work is the development and demonstration of RAGnRoll, a comprehensive benchmarking framework designed to facilitate systematic and reproducible RAG evaluation.

RAGnRoll, detailed in Chapter \ref{chap:design}, provides a structured methodology to navigate the intricacies of RAG development. Key features of the framework include:
\begin{itemize}
    \item \textbf{Systematic Baselining:} The framework mandates the evaluation of standalone LLMs and naive RAG configurations as baselines (as argued in Section \ref{sec:e2e}). This crucial step helps quantify the added value of more complex RAG architectures and informs the decision of whether RAG is beneficial for the specific task and dataset.
    \item \textbf{Rigorous Validation and Generalization Assessment:} Inspired by standard machine learning practices, RAGnRoll enforces a strict validation-test split of evaluation data (Section \ref{sec:valtestsplit}). All system tuning and iterative reconfiguration are performed exclusively on the validation set, with the test set reserved for a final assessment of generalization error, thereby mitigating the risk of overfitting to the development data, a critical concern when tuning complex RAG systems.
    \item \textbf{Comprehensive Multi-faceted Evaluation:} Recognizing that RAG performance is multi-dimensional, the framework supports both end-to-end evaluation and granular component-level analysis. This allows for the identification of bottlenecks within the RAG pipeline. Furthermore, this thesis argues for and the framework design accommodates the inclusion of hardware metrics (e.g., latency, resource utilization) alongside traditional accuracy-based metrics, providing a holistic view of a system's practicality.
    \item \textbf{Transparent and Reproducible Experimentation:} Leveraging tools like MLflow for results logging and Langfuse for detailed execution tracing (Section \ref{sec:ui}), RAGnRoll promotes transparency and aids in in-depth failure analysis. The use of Haystack's declarative pipeline configurations further enhances reproducibility.
\end{itemize}

The practical utility and effectiveness of the RAGnRoll framework were demonstrated in Chapter \ref{chap:Experiment} through an experiment focused on software configuration validation. This application showcased how the framework can guide the iterative development and comparison of different RAG systems, leading to optimized configurations tailored to specific requirements, including considerations for output formatting and computational overhead.

The practical application in Chapter \ref{chap:Experiment} served to demonstrate RAGnRoll's core functionalities. However, the breadth of RAG configurations explored in that instance was necessarily guided by the project's timeframe and resources. The reported experiments, with an approximate cost of 200 Euro, highlight the framework's utility for initial, cost-effective investigations. A key strength of RAGnRoll is its design for scalability; with more extensive time and financial backing, it can be readily employed to systematically evaluate a far wider spectrum of embedding models, generation models, and intricate RAG architectures than was feasible within this thesis's direct experimental scope. This underscores its potential for larger-scale research or industrial deployment.

In conclusion, this thesis provides both a conceptual blueprint and a practical implementation for the robust evaluation of RAG systems. By emphasizing systematic comparison, rigorous testing for generalization, comprehensive metric collection, and detailed failure analysis, the RAGnRoll framework empowers researchers and practitioners to make data-driven decisions in the rapidly evolving landscape of retrieval-augmented generation. It offers a path to move beyond ad-hoc tuning towards a more principled approach to building effective and efficient RAG solutions, ultimately helping to answer whether a RAG system is indeed necessary and how to best configure it for a specific use-case.