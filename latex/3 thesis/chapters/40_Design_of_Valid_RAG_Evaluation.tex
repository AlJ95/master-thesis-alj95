% 2. RAG Entwicklung, welche Frameworks gibt es 
% -> Konfigurierbarkeit durch Files
% -> Maintainability
% -> Beliebtheit?
% -> Future Work (Haystack UI)
% -> Modularität (nach Gao.)


Developing retrieval-augmented generation systems is a difficult task that needs several reconfiguration phases.\cite{Simon.10112024} Component evaluation and in-depth failure analysis are key requirements for readjusting the right parts within the RAG-system, as they can include complex pipelines that may involve iterative or recursive processes. Failures can occur in many parts of the RAG-system. Barnett et al.\cite{Barnett.2024} created a list of 7 failures occuring in an advanced RAG system with Rewrite-Retrieve-Rerank-Read (4R) structure as Gao et al.\cite{Gao.18.12.2023} has defined it. Therefore for a framework it is indespensible to evaluate all components next to the end-to-end evaluation for the overall result. In this chapter, we will discuss how RAGs are evaluated accurately in two different aspects - end-to-end evaluation and component evaluation. After that we will discuss how fast RAG development with transparent and reproducible results can be done. We will introduce Haystacks approach of fast modular RAG development. Lastly we will introduce the novelle blueprint by Simon et al.\cite{Simon.10112024} for handling external validity in RAG experiments. 

\section{Evaluation Techniques}

Evaluating retrieval-augmented generation systems is a hard task that is still highly researched. It comes with all machine learning based evaluation problems such as data shifts, generalization errors or data contamination, but has also a variety of failure points introduced by the design of such a complex system. In this section I will introduce a majority of failure points for RAGs and present, how this framework helps to identify them. The final goal of tuning a RAG system is always to maximize its performance in end-to-end evaluation. RAG users want that the response of the systems fully answers its question or completes its task correct. Therefore end-to-end evaluation is important for comparing baselines and thus to decide if a RAG system can be a improvement to standalone large language models. 


The problem that occurs with end-to-end evaluation is that it is far from obvious which parameters we have to tune or which data we have to provide to accomplish a performance improvement. In figure \ref{fig:failures} there is an example for every component in particular RAG system. In reality there are much more points of failures in such systems. We will explore those in section component evaluation.

\begin{figure}
  \centering
  \includegraphics[width=\textwidth]{images/FailurePointExamples.pdf}
  \caption{A RAG pipeline with one failure example of all used components.}
  \label{fig:failures}
\end{figure}

\subsection{End-to-End Evaluation}

We define end-to-end evaluation as metrics that use input and actual output to compare against a ground-truth output to decide if the evaluated item was correct or not. Evaluation that address retrieval are therefore not part of it. In this thesis we will focus on classification tasks only. Therefore the metrics we use for end-to-end evaluations are also limited by classification ones. 

{\renewcommand{\arraystretch}{1.5}%
\begin{table}
  \centering
 \begin{tabular}{|l|l|}
  \hline
  \textbf{Metric} & \textbf{Formula / Description} \\[3pt]
  \hline Accuracy & $\frac{TP + TN}{TP + TN + FP + FN}$\\[5pt]
  \hline Precision & $\frac{TP}{TP + FP}$\\[5pt]
  \hline Recall & $\frac{TP}{TP + FN}$\\[2pt]
  \hline F1-Score & $2 \times \frac{Precision \times Recall}{Precision + Recall}$\\[2pt]
  \hline Matthews Correlation & $\frac{TP \times TN - FP \times FN}{\sqrt{(TP + FP)(TP + FN)(TN + FP)(TN + FN)}}$\\Coefficient & \\[2pt]
  \hline False-Positive Rate & $\frac{FP}{FP + TN}$\\[2pt]
  \hline False-Negative Rate & $\frac{FN}{FN + TP}$\\[2pt]
  \hline Receiver Operating & Curve plotting TPR against FPR at various \\&threshold settings\\[2pt]
  \hline Area under ROC Curve &  AUC of the ROC curve\\[2pt]
  \hline pass@k & Probability of a correct solution in \textit{k} attempts \\[2pt]
  \hline
 \end{tabular}
 \caption{Typical classification metrics used for experiments including RAGs or LLMs\cite{Hou.8212023,Zeng.28.03.2024}}
 \label{table:classification_metrics}
\end{table}}

In stark contrast to evaluating generative tasks, classification comes with clear metrics or methods to evaluate outputs. In table \ref{table:classification_metrics} there is a list of highly used classification metrics for RAG and LLM experimentation.\cite{Hou.8212023,Zeng.28.03.2024} 

End-to-end evaluation requires the researcher to establish a baseline for the experiments. Baslines make experiment results interpretable and ensure that there is something to compare with. Therefore we decided to implement two baselines in this framework, which are used per default if an experiment starts. In the first run, we use the selected generator model alone to get a baseline vs standalone LLMs. Next we run a naive RAG with the data from the predefined corpus, the basic retrieval and generator (retrieve-read, cf: section \ref{sec:naive_rags}). The standalone LLM baseline ensure that the complexity overhead of implementing a RAG system is justified. If the evaluated RAG system can not surpass the performance of the LLM baseline, then simpler system should be used. Similar to that is the naive RAG baseline used to compare more complex RAG variations with more overhead, computation time or costs to the simpler naive one to ensure that the mentioned downsides are justified. 

\begin{figure}[h]
  \centering
  \includegraphics[width=\textwidth]{images/FrameworkBaselines.pdf}
  \caption{Framework: research define configuration files of RAG variations to evaluate them against the baselines standalone LLM and naive RAG with end-to-end evaluation.}
  \label{fig:framework-baselines}
\end{figure}
% \cite{Ru.15.08.2024.} Eventhough all components of a RAG-system affect its overall performance, there are ones that can not be evaluated directly.  \\

We illustrated the prior defined state of our framework in figure \ref{fig:framework-baselines} to enhance the understanding of our considerations. It will be updated with each functionality. It visualizes the process of using our evaluation data to calculate metrics of our baselines next to the configurations we want to test in the experiment. It shows that the evaluation is done by end-to-end as well as component-wise for generation and retrieval. In the following section we will explain, how component evaluation works in our framework.


\subsection{Component Evaluation}

Component evaluation is necessary to detect performance issues of the whole systems introduced through individual components or blocks of components that are connected and do not work well together.\cite{Salemi.2024} In figure \ref{fig:failures}, we presented few of them. While this gives a good first impression of the variety of failures in a RAG system, this visualisation is far from complete. We argue that all components, no matter how small their impact seem, can introduce significant failure rate for the overall system. Therefore it would be necessary to evaluate them all component-wise. 

We are not aware of papers that analyze component evaluation beside retrieval and read - the core components of a RAG system. We also want to make clear, that evaluating components fully isolated by other components is not practical. If we take for example the generator component and want to evaluate it completely apart from retrieval, then we must provide a ground-truth dataset of retrieval chunks. This is doable for easy scenarios in retrieval that need one document chunk e. g. a sentence where a query is exactly answered, but in real-world scenarios there are much more complex cases such as ambigious queries, complex queries or queries that won't need retrieval at all.\cite{Huang.2023} If the researcher adjusts chunk size or technique, then he must also create a new set of ground-truths as the generated documents have changed and the previous selected ones do not exist anymore (cf. section \ref{sec:advanced_rags}). 

If we consider other components then retriever such as rewrite or rerank then there is no ground truth of data. The purpose of rewriting the query is increasing the following retrieval. Therefore the metric is dependend on the retrieval technique too can not be fixed ground truth examples in an evaluation dataset.

The purpose of component evaluation is to find weak spots within the RAG. It is therefore more feasible to evaluate a RAG-system only with system inputs and its ground truth final answers and let a non-deterministic LLM-as-a-Judge evaluate for steps within the RAG-system. LLM-as-a-Judge is a evaluation techniques, where a significantly more performant model or solely on evaluating trained model does the evaluation. Instead of real-valued metrics, an LLM decides and asseses the solution of the task. 

Relying only on system input and ground-truth datasets for evaluation saves a lot of time. Resulting failure analysis of the system and its components must nevertheless rely on tracing single pipeline runs down, which we will discuss in a latter section. In the following sections we will present our evaluation of the retrieval block and the generation block.


\paragraph{Retrieve}
Retrieval Evaluation is possibly the most difficult part in the evaluation for several reasons. First, more retrieval do not imply better results and it is not clear nor obvious which set of documents perfectly serves the generator to give the best answer.\cite{Jin.5222024} Second, in a real world scenario indexed corpus, there are redundant and contradicting information.\cite{Yu.2024} Third, some questions or queries require to view the topic in a diverse manner, highlighting several considerations for different perspectives and fourth, the quality of the query does also affect the retrieval quality, because queries can be too complex and ambigious or not even requiring retrieval, which may also lead to LLM hallucinations.\cite{Huang.2023, Mallen.20.12.2022}

Retrieved documents might contain useful information or not and are therefore binary assessable, enabling metrics that handle \textit{True-Positives, True-Negatives, False-Positives} and \textit{False-Negatives}. However, simple metrics such as recall and precision are not suited to measure the small differences in retrieval.\cite{Yu.2024} Additionally there are limitations while working with ground-truths in retrieval evaluation. Ground truth context for a given query might be ambiguously. Selecting a ground-truth set of relevant information is a task that even expert humans are not able to solve perfectly and LLM-as-a-Judge models show high correlation with humans.\cite{Chiang.2023} Therefore we will focus in this framework solely on LLM-as-a-Judge evaluation for retrieval. 

One metric for retrieval is context relevance, which is a LLM-as-a-Judge metric with the goal to decide for each document, if it is important regarding a certain query. 

\textcolor{blue}{Finde noch weitere Metriken dafür. RAGAS, Context Precision, ....}


\paragraph{Generators}
Generators are based on transformer architecture and classification tasks were not the initial intention for this kind of tasks. Thus it is important that outputs are in a wanted form such as \textit{valid}, or \textit{invalid}. Therefore we need a metric that evaluates the generators ability to follow those rules. We call this \textit{format-checker}. The problem that comes with this evaluation is that it is impossible to check context utilization, answer relevance and so on. Evaluating generator just with classification metrics and \textit{format-checker} would be indistinguishable with end-to-end evaluation. Therefore we will use a slightly different format for the classification task. Instead of generating only binary outputs such as \textit{0} / \textit{1}, \textit{valid} / \textit{invalid} or \textit{True} / \textit{False}, we will also output the reason for this decision afterwards. Then we can evaluate the generator component just as other components with commonly used metrics such answer relevance next to classification metrics for our downstream task. This procedure is required for failure analysis and therefore for improving the system in the reconfiguration phase. A binary output lacks the information we need to decide, which information the generator could not utilize. 

\textcolor{blue}{Finde ein gutes Beispiel dafür in meinem SE Experiment. Ziel ist es zu zeigen, dass ein simples "Ja/Nein" nicht zeigt, warum der Generator die Frage falsch beantwortet hat, welches aus einer binären Antwort nicht hervorgeht.}




\subsection{Component Block Evaluation}

After evaluating component-wise, we need to evaluate other components, that are not assessable isolated too. In an advanced RAG such a component would be the rewriter. Rewriters try to synergize query to retrieval and context. Defining a performant rewriter depends on whether the RAG uses a sparse or a dense retrieval as he must either use relevant keywords or simulate semantic similarity. The second component from an advanced RAG system would be reranking. Both, reranker and generator have an great impact and interest in context utilization, the ability to build a coherent and correct answer based on actual facts from the retrieved context. Phenomena such as lost-in-the-middle have a significant affect on context utilization, but LLMs differ in this effect.\cite{Liu.06.07.2023} Therefore it might happen that the best reranker and generator regarding their component-wise performance lead to inferior context utilization results. The system needs to be model-agnostic for all components and researchers need rigorous testing of potential component blocks. 

\paragraph{How to evaluate such component blocks?}

\begin{figure}
  \centering
  \includegraphics[width=\textwidth]{images/ComponentBlockEvaluation.pdf}
  \caption{Comparison of isolated component evaluation of each component and component block evaluation, where several components with the same goal are evaluated.}
  \label{fig:componentblockeval}
\end{figure}

\textcolor{blue}{Abbildung neumachen.}

We differentiate two major blocks - pre-retrieval and post-retrieval. Evaluating pre-retrieval component blocks such as rewriter is a derived task from the retrieval evaluation. The goal of pre-retrieval is to increase the retrieval quality so that the retriever finds all necessary documents or chunks for this task. Commonly used pre-retrieval techniques are query routing, query transformation and query expansion. Even in the ingestion stage, chunking, document selection and preprocessing steps are tuning parameters that influence retrieval results and its quality. 

Post-retrieval components such as reranker aim to improve the context utilization for the generator. The generator requires a curated and limited list of documents that serves its ability and query best. Second, reducing the Top-K parameter for retrievers might result in missing documents with required information. For that we use the same metrics as for the retriever. The additional step of reranking influences the overall performance and evaluating this component is important. 

Most components beside retriever and generator have no direct measurement,there they have to be evaluated by replacement and metrics of the component block. For example, the rewriter is evaluated by using similar components, different models or prompting techniques as rewriter-variations. Then the RAG runs through the evaluation by only changing those rewriter specific configurations. The resulting end-to-end or if applicable the resulting retrieval metrics decide which rewriter setting is better suited.

This requires the researcher to try many different RAG variations for rigorous testing. In the next section we will introduce our appraoch for fast RAG development.

\section{Tracing}

\section{Fast RAG Development}

\begin{figure}[b]
    \centering
    \includegraphics[width=\textwidth]{images/showcase-pipeline.pdf}
    \caption{A simple Retrieve-Read pipeline with both dense and sparse retrieval.}
    \label{fig:showcase}
\end{figure}


There are several RAG development tools and frameworks. Highly used ones are Llama-Index\cite{Liu_LlamaIndex_2022}, Langchain\cite{Chase_LangChain_2022} and Haystack\cite{Pietsch_Haystack_the_end-to-end_2019}. All of them offer comparable functionality for building advanced RAG systems in modularized architecture as introduced by Gao et al.\cite{Gao.18.12.2023}. Haystack does offer a functionality, where users can define a pipeline via a YAML-file. This enhances the reconfiguration of such systems, because instead of editing python files, there are just parameters in one YAML-file to be changed. This enhances the ability to report the methodologies of each experiment too. Instead of saving a python script or module for each configuration, only a YAML is stored that describes the tested RAG architecture fully. An example of this YAML definition can be seen in figure \ref{fig:showcase} and the YAML code below.

There are are also two more benefits for configuring via YAML files. First, we can copy existing RAG configurations easily and second, Haystack develops an UI for developing RAG system, that can be used to manage even very complex systems on a two-dimensional space. A demo screenshot can be seen on figure \ref{fig:deepsetstudio}. This simplifies RAG system development furthermore.


\begin{figure}[b]
  \centering
  \includegraphics[width=\textwidth]{images/deepset-studio-zoomed-in.png}
  \caption{Haystack's Deepset Studio application for drag and drop RAG development.\cite{deepsetstudio.10.03.2025}}
  \label{fig:deepsetstudio}
\end{figure}


\newpage
\begin{minted}[
    frame=single,
    bgcolor=lightgray
  ]{yaml}
components:
  llm:
    init_parameters:
      api_base_url: null
      api_key:
        env_vars: OPENAI_API_KEY
      ...
  prompt_builder: ...
  bm25_retriever: ...
  embedding_retriever: ...
  joiner: ...
  text_embedder: ...
  docs_embedder: ...
connections:
- receiver: llm.prompt
  sender: prompt_builder.prompt
- ...
\end{minted}

\textcolor{blue}{Soll ich hier noch über Custom Components sprechen?}


\section{Reconfiguration}
The visualisation shows the process of reconfiguring the RAG system till its optimized towards a specific validation set. The reconfiguration phase can adjust parameters, using different models, pipelines or even fine-tuning parts such as retriever or generator. We will introduce detailed validity concerns of this common approach in a latter section. 

\section{Reporting}

\section{Increasing Transparency in RAG Experiments}

\begin{itemize}
    \item DVC for RAG Experiments
    \item Save YAMLs, Versionize Data + Code
\end{itemize}

\section{Internal Validity}

\textcolor{blue}{Recherche über Validity (Internal vs External)}


Evaluating semantics in LLM's or RAG systems rely on LLM-as-a-Judge models or lexical metrics such as BLEU or ROUGE, which measure the token overlap of actual answer and ground truth. Lexical answers can not capture answer pairs that are differently written but having the same semantic. LLM-as-a-Judge 

\paragraph{Preference Leakage}
There are findings suggesting that LLM-as-a-judge favor similar LLMs that their based on. Therefore using synthetic data for evaluation is fast, but risks so called preference leakage.\cite{Li.03.02.2025} This field is very new and still under research. 


\paragraph{RAG Evaluation needs a Holdout-Testset}
\begin{itemize}
    \item RAG Experiments belongs to ML Experiments 
    \item Requires Seeding
    \item ML Experiments might cause Overfitting
    \item Generalization (External Validity) is testable with a hard data split (train vs validation vs test)
    \item training data split is not needed, but validation and test is required
    \item validation is for hyperparameter tuning important, which configs have the best results
    \item no testset -> no generalization estimatation
    \item \textcolor{red}{Problem zum Nachdenken} Im Training braucht es ein Holdout-Testset damit man das LLM auswählen kann, welches den geringsten Testfehler hat, aber im RAG wird nicht trainiert. Müssen dann alle Konfigurationen dagegen getestet werden?
\end{itemize}

\section{External Validity}

\begin{figure}
    \centering
    \includegraphics[width=\textwidth]{images/RAGvsLLM-tuning.pdf}
    \caption{Comparison of reconfiguration between RAGs and LLMs - both relying on tuning parameters and test data.}
    \label{fig:tuning}
\end{figure}

% Misconception? RAGs don't need Test-Dataset, because we do not train? But Overfitting is not dependend on training perse, but on free parameter, which happes in RAGs too. Just because we do not have Millions up to Billions of Parameters, does not mean we don't have to consider overfitting. <calculate number of parameters> 


We demand for a test data set to avoid overfitting while tuning RAG parameter. Data is rare, especially for down-stream tasks in domains such as configuration validation or similar tasks. They often rely on few datasets that can not represent the whole domain. Therefore there are measures required to ensure that the developed systems can be generalized from the seen data to the unseen production data. Measuring generalization error in the large language model development is done by splitting the data into training, validation and test datasets. 

While the training and validation data is used for hyperparameter tuning to ensure the best performing model, the test dataset on the other hand is used to check if the hyperparameter tuning introduced an overfitting. This procedure must be adopted for RAG systems too (cf. figure \ref{fig:tuning}). Even though they do not rely on a training per se, the continous reconfiguration of such systems, including prompt engineering or parameter tuning can lead to overfitting. The needs for a test dataset is independent from a potential training phase, but instead very dependent on a tuning an high amount of free variables, that is done till the results converge to the best outcoming scenario.

Both LLM and RAG development share those reconfiguration phases. Additionally there are situations when individual components such as retriever must be fine-tuned to handle domain-specific data. In such situations there should not only be test data for the retrieval training part, but also for the overall RAG system to ensure overfitting did not happen on the bigger system.


\section{User Interface}

\begin{figure}[!ht]
    \centering
    \includegraphics[width=\textwidth]{images/Sketch.pdf}
    \caption{...}
    \label{fig:EvaluationDesign}
\end{figure}

We use the presented Haystack functionalities to build a CLI-based framework around it. Our complete appraoch can be seen in figure \ref{fig:EvaluationDesign}. Given a data directory, pipeline and metrics definition, we first split the data into a validation and holdout-test dataset based on a split parameter \textit{test\_size}. Next we use the resulting validation dataset to run the first evaluations on the data. For that it loads the pipeline and data from its paths and starts with evaluating against a standalone LLM to have a baseline if RAG at all is a significant improvement in contrast to having just an LLM. Next, it ingests data into a vector database and builds an naive RAG with (Retrieve-Read) architecture as another baseline in contast to the in the YAML file defined advanced RAG architecture. The same procedure starts with the RAG configuration from the file. Each run gets evaluated in respect to End-to-End evaluation and component-wise. Evaluations are visualized in \textcolor{red}{EVAL INFOS EINFÜGEN Framework oder PRozess für Indepth Failure Analysis hier erklären}. After failure analysis, the RAG can be reconfigured again to test new parameters. 

\textcolor{blue}{Here comes text how to use Holdout-Test dataset}



\section{Limitations}

List of possible Limitations

\begin{itemize}
    \item Generation Tasks 
    \item Chunking Reconfiguration
\end{itemize}